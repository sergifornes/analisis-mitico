\documentclass[letterpaper,11pt,twocolumn,twoside,]{pinp}

%% Some pieces required from the pandoc template
\providecommand{\tightlist}{%
  \setlength{\itemsep}{0pt}\setlength{\parskip}{0pt}}

% Use the lineno option to display guide line numbers if required.
% Note that the use of elements such as single-column equations
% may affect the guide line number alignment.

\usepackage[T1]{fontenc}
\usepackage[utf8]{inputenc}

% pinp change: the geometry package layout settings need to be set here, not in pinp.cls
\geometry{layoutsize={0.95588\paperwidth,0.98864\paperheight},%
  layouthoffset=0.02206\paperwidth, layoutvoffset=0.00568\paperheight}

\definecolor{pinpblue}{HTML}{185FAF}  % imagecolorpicker on blue for new R logo
\definecolor{pnasbluetext}{RGB}{101,0,0} %



\title{Análisis de Personajes Mitológicos}

\author[]{Sergi Fornés}


\setcounter{secnumdepth}{0}

% Please give the surname of the lead author for the running footer
\leadauthor{}

% Keywords are not mandatory, but authors are strongly encouraged to provide them. If provided, please include two to five keywords, separated by the pipe symbol, e.g:
 \keywords{  mitología |  web scraping |  wikipedia |  natural language
processing |  network analysis  }  

\begin{abstract}
El objetivo del trabajo es crear una base de datos de personajes
mitológicos a partir de
\href{https://es.wikipedia.org/wiki/Wikipedia:Portada}{Wikipedia}. Una
vez hecho esto, se pretenden analizar los artículos de estos personajes
y las relaciones entre ellos haciendo uso de algoritmos de Procesamiento
de Lenguaje Natural (NLP) y Network Analysis.
\end{abstract}

\dates{Trabajo Final de Máster}

\documentdate{Guión}

% initially we use doi so keep for backwards compatibility
% new name is doi_footer

\pinpfootercontents{Sergi Fornés}

\begin{document}

% Optional adjustment to line up main text (after abstract) of first page with line numbers, when using both lineno and twocolumn options.
% You should only change this length when you've finalised the article contents.
\verticaladjustment{-2pt}

\maketitle
\thispagestyle{firststyle}
\ifthenelse{\boolean{shortarticle}}{\ifthenelse{\boolean{singlecolumn}}{\abscontentformatted}{\abscontent}}{}

% If your first paragraph (i.e. with the \dropcap) contains a list environment (quote, quotation, theorem, definition, enumerate, itemize...), the line after the list may have some extra indentation. If this is the case, add \parshape=0 to the end of the list environment.


\hypertarget{guiuxf3n}{%
\section{Guión}\label{guiuxf3n}}

Entender la mitología es entender al ser humano. Diferentes culturas han
creado mitos para responder cuestiones complejas de manera sencilla. En
estos mitos aparecen personajes sobrenaturales y las relaciones que
mantienen con los seres humanos.

En los mitos aparecen todo tipo de personajes, desde diosas del amor
hasta monstruos devoradores de humanos. Dada esta diversidad aparecen
preguntas interesantes, ¿diferentes culturas tienen personajes
similares? ¿existen relaciones entre las características de diferentes
personajes? ¿y entre las características de los mismos personajes?

Los objetivos de este trabajo son:

\begin{enumerate}
\def\labelenumi{\arabic{enumi}.}
\tightlist
\item
  Crear un conjunto de datos de personajes mitológicos de diversas
  culturas.
\item
  Analizar el conjunto de datos, principalmente el texto de los
  personajes mitológicos.
\end{enumerate}

\hypertarget{creaciuxf3n-del-conjunto-de-datos}{%
\subsection{Creación del conjunto de
datos}\label{creaciuxf3n-del-conjunto-de-datos}}

El conjunto de datos se obtendrá utilizando librerías de web-scraping
(\texttt{Beautiful\ Soup} y \texttt{Selenium}) de Python.

Todos los datos se extraerán de la enciclopedia digital
\href{https://wikipedia.es}{Wikipedia}. Hay que señalar que cualquier
persona puede editar esta enciclopedia, por lo que es posible que el
conjunto de datos tenga errores. Sin embargo, Wikipedia es una de las
plataformas de contenido libre más completas de internet.\\
Existen artículos en Wikipedia de prácticamente todos los personajes
mitológicos. Estos artículos están clasificados en categorías similares,
por lo que se navegará a través de estas categorías para obtener el
texto disponible de los personajes. Además se añadirá información
adicional de cada personaje en función de la categoría en la que se
encuentre, como la mitología, la región, el continente, o el tipo de
personaje (deidad, héroe, monstruo,\ldots).

Por otro lado, los artículos de los personajes mitológicos están
relacionados entre sí mediante enlaces, así que también se extraerá una
matriz con estas relaciones.

\hypertarget{anuxe1lisis-del-conjunto-de-datos}{%
\subsection{Análisis del conjunto de
datos}\label{anuxe1lisis-del-conjunto-de-datos}}

Se realizará un análisis exploratorio de los datos y se usarán técnicas
de análisis de grafos sobre la matriz de relaciones para crear un índice
de importancia de los personajes.

La variable más importante de los personajes es el texto extraído de
Wikipedia, así que también se aplicarán diversos algoritmos de NLP para
su análisis.

%\showmatmethods





\end{document}
