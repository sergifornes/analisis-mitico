\documentclass[letterpaper,9pt,twocolumn,twoside,]{pinp}

%% Some pieces required from the pandoc template
\providecommand{\tightlist}{%
  \setlength{\itemsep}{0pt}\setlength{\parskip}{0pt}}

% Use the lineno option to display guide line numbers if required.
% Note that the use of elements such as single-column equations
% may affect the guide line number alignment.

\usepackage[T1]{fontenc}
\usepackage[utf8]{inputenc}

% pinp change: the geometry package layout settings need to be set here, not in pinp.cls
\geometry{layoutsize={0.95588\paperwidth,0.98864\paperheight},%
  layouthoffset=0.02206\paperwidth, layoutvoffset=0.00568\paperheight}

\definecolor{pinpblue}{HTML}{185FAF}  % imagecolorpicker on blue for new R logo
\definecolor{pnasbluetext}{RGB}{101,0,0} %



\title{Herramienta para el Análisis de Personajes Mitológicos}

\author[]{Sergi Fornés}


\setcounter{secnumdepth}{0}

% Please give the surname of the lead author for the running footer
\leadauthor{Author}

% Keywords are not mandatory, but authors are strongly encouraged to provide them. If provided, please include two to five keywords, separated by the pipe symbol, e.g:
 \keywords{  one |  two |  optional |  keywords |  here  }  

\begin{abstract}
Your abstract will be typeset here, and used by default a visually
distinctive font. An abstract should explain to the general reader the
major contributions of the article.
\end{abstract}

\dates{This version was compiled on \today} 


% initially we use doi so keep for backwards compatibility
% new name is doi_footer
\doifooter{\url{https://cran.r-project.org/package=YourPackage}}

\pinpfootercontents{YourPackage Vignette}

\begin{document}

% Optional adjustment to line up main text (after abstract) of first page with line numbers, when using both lineno and twocolumn options.
% You should only change this length when you've finalised the article contents.
\verticaladjustment{-2pt}

\maketitle
\thispagestyle{firststyle}
\ifthenelse{\boolean{shortarticle}}{\ifthenelse{\boolean{singlecolumn}}{\abscontentformatted}{\abscontent}}{}

% If your first paragraph (i.e. with the \dropcap) contains a list environment (quote, quotation, theorem, definition, enumerate, itemize...), the line after the list may have some extra indentation. If this is the case, add \parshape=0 to the end of the list environment.

\acknow{This template package builds upon, and extends, the work of the
excellent \href{https://cran.r-project.org/package=rticles}{rticles}
package, and both packages rely on the
\href{http://www.pnas.org/site/authors/latex.xhtml}{PNAS LaTeX} macros.
Both these sources are gratefully acknowledged as this work would not
have been possible without them. Our extensions are under the same
respective licensing term
(\href{https://www.gnu.org/licenses/gpl-3.0.en.html}{GPL-3} and
\href{https://www.latex-project.org/lppl/}{LPPL (\textgreater= 1.3)}).}

\hypertarget{guiuxf3n}{%
\subsection{Guión}\label{guiuxf3n}}

Entender la mitología es entender al ser humano. Diferentes culturas han
creado mitos para responder cuestiones complejas de manera sencilla. En
estos mitos aparecen personajes sobrenaturales y las relaciones que
mantienen con los seres humanos.

En los mitos aparecen todo tipo de personajes, desde diosas del amor
hasta monstruos devoradores de humanos. Dada esta diversidad aparecen
preguntas interesantes, ¿diferentes culturas tienen personajes
similares? ¿existen relaciones entre las características de los
personajes? ¿y entre los mismos personajes?

Los objetivos de este trabajo son:

\begin{enumerate}
\def\labelenumi{\arabic{enumi}.}
\tightlist
\item
  Crear un conjunto de datos de personajes mitológicos de diversas
  culturas.
\item
  Analizar el conjunto de datos.
\item
  Crear una herramienta de análisis de personajes mitológicos.
\end{enumerate}

\hypertarget{creaciuxf3n-del-conjunto-de-datos}{%
\subsubsection{Creación del conjunto de
datos}\label{creaciuxf3n-del-conjunto-de-datos}}

El conjunto de datos se obtendrá utilizando herramientas de web-scraping
en Python.

Todos los datos se extraerán de la enciclopedia digital
\href{https://wikipedia.es}{Wikipedia}. Hay que señalar que cualquier
persona puede editar esta enciclopedia, por lo que es posible que el
conjunto de datos tenga errores. Sin embargo, Wikipedia es una de las
plataformas de contenido libre más completas de internet.\\
Existen artículos en Wikipedia de prácticamente todos los personajes
mitológicos. Estos artículos están clasificados en categorías similares,
por lo que se navegará a través de estas categorías para obtener el
texto disponible de los personajes. Además se añadirá información
adicional de cada personaje en función de la categoría en la que se
encuentre.

Por otro lado, los artículos pueden tener enlaces a artículos de otros
personajes, así que también se extraerá una matriz con estas relaciones.

\hypertarget{anuxe1lisis-del-conjunto-de-datos}{%
\subsubsection{Análisis del conjunto de
datos}\label{anuxe1lisis-del-conjunto-de-datos}}

Se realizará un análisis exploratorio de los datos y se usarán técnicas
de análisis de grafos sobre la matriz de enlaces para crear un índice de
importancia de los personajes.

La variable más importante de los personajes es el texto extraído de
Wikipedia, así que también se aplicarán diversas técnicas de análisis de
texto. Así mismo se aplicarán técnicas de análisis de grafos sobre las
conclusiones del análisis de texto.

\hypertarget{aplicaciuxf3n-para-la-creaciuxf3n-de-personajes}{%
\subsubsection{Aplicación para la creación de
personajes}\label{aplicaciuxf3n-para-la-creaciuxf3n-de-personajes}}

Se creará una herramienta interactiva con la que poder analizar
personajes o mitologías de diversas maneras. Por ejemplo, a partir de la
palabra ``mar'', se podrá obtener un conjunto de características comunes
a personajes mitológicos relacionados con el mar.

%\showmatmethods
\showacknow


\bibliography{pinp}
\bibliographystyle{jss}



\end{document}

